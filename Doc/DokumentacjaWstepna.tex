\documentclass{scrartcl}
\usepackage{polski}
\usepackage[utf8]{inputenc}
\usepackage{amsmath}
\usepackage{listings}
\usepackage{amsmath}

\usepackage{algorithm}
\usepackage[noend]{algpseudocode}

\makeatletter
\def\BState{\State\hskip-\ALG@thistlm}
\makeatother

\makeatletter
\def\BState{\State\hskip-\ALG@thistlm}
\makeatother


\title{Dokumentacja wstępna}
\subtitle{Implementacja i testowanie algorytmu ewolucji różnicowej, w którym jako pierwszy z 3 punktów stosowanych podczas mutacji, wybierana jest średnia punktów populacji. Metodę należy porównać z klasyczną wersją ewolucji różnicowej. Testy powinny zostać przeprowadzone na benchmarku CEC 2013}
\date{\today}

\author{
  Michał Padzik\\
  \texttt{padzikm@student.mini.pw.edu.pl}
  \and
  Albert Wolant\\
  \texttt{wolanta@student.mini.pw.edu.pl}
}

\begin{document}

\maketitle
\pagenumbering{gobble}
\newpage
\pagenumbering{arabic}

\section{Opis algorytmów}

\subsection{Klasyczna ewolucja różnicowa}

Poniżej zamieszczamy pseudokod klasycznego algorytmu ewolucji różnicowej, który zostanie zaimplementowany w celu porównania z algorytmem zmodyfikowanym:
\vspace{5mm}

\begin{algorithm}
\caption{Klasyczny algorytm ewolucji różnicowej}\label{euclid}
\begin{algorithmic}
\Procedure{differential evolution}{}

\State $P^0 \gets \{ P^0_1, P^0_2, ... , P^0_n\}$
\State $H \gets P^0$
\State $t \gets 0$

\While{!stop} 
\ForAll{$i  \in \{1 : n\}$}
\State $P^t_j \gets select(P^t)$
\vspace{1mm}
\State $P^t_k, P^t_l \gets sample(P^t)$
\vspace{1mm}
\State $M^t_i \gets P^t_j + F * (P^t_k - P^t_j)$
\vspace{1mm}
\State $O^t_i \gets crossover(P^t_i, M^t_i)$
\vspace{1mm}
\State $H \gets H \cup \{O^t_i\}$
\vspace{1mm}
\State $P^{t+1}_i \gets tournament(P^t_i, O^t_i)$
\vspace{1mm}
\EndFor
\State $t \gets t+1$
\EndWhile

\EndProcedure
\end{algorithmic}
\end{algorithm}

Operacja $tournament$ to wybór osobnika do kolejnej populacji, a zbiór $H$ jest zbiorem kandydatów do populacji. Jeśli przyjmiemy, że operacja $select$ jest losowym wyborem pary punktów z jednakowym prawdopodobieństwem, a operacja $crossover$ to krzyżowanie wymieniające, dwumianowe to przedstawiony powyżej algorytm nosi w literaturze oznaczenie $DE/rand/1/bin$ i jest najpopularniejszym wariantem algorytmu ewolucji różnicowej, dlatego zostanie użyty do porównań jako klasyczny przykład ewolucji różnicowej.

\pagebreak
\subsection{Zmodyfikowana ewolucja różnicowa}

Proponowana modyfikacja algorytmu ewolucji różnicowej właściwie obejmuje tylko operację $select$. Zostanie ona zmieniona na operację wyliczania średniej z aktualnej populacji. Dodatkowo, ponieważ wartość średniej nie zmieni się dla kolejnych osobników w populacji może być obliczona tylko raz dla wszystkich. Poniżej przedstawiamy pseudokod zmodyfikowanego algorytmu:

\vspace{5mm}

\begin{algorithm}
\caption{Zmodyfikowany algorytm ewolucji różnicowej}\label{euclid}
\begin{algorithmic}
\Procedure{modified differential evolution}{}

\State $P^0 \gets \{ P^0_1, P^0_2, ... , P^0_n\}$
\State $H \gets P^0$
\State $t \gets 0$

\While{!stop} 
\State $A \gets average(P^t)$
\ForAll{$i  \in \{1 : n\}$}
\vspace{1mm}
\State $P^t_k, P^t_l \gets sample(P^t)$
\vspace{1mm}
\State $M^t_i \gets A + F * (P^t_k - P^t_j)$
\vspace{1mm}
\State $O^t_i \gets crossover(P^t_i, M^t_i)$
\vspace{1mm}
\State $H \gets H \cup \{O^t_i\}$
\vspace{1mm}
\State $P^{t+1}_i \gets tournament(P^t_i, O^t_i)$
\vspace{1mm}
\EndFor
\State $t \gets t+1$
\EndWhile

\EndProcedure
\end{algorithmic}
\end{algorithm}

Operacja $average$ wyliczy średnią z populacji.

\section{Opis eksperymentu}

Eksperymenty będą polegały na porównaniu wyników działania algorytmów klasycznego i zmodyfikowanego, na benchmarku CEC-2013. Dodatkowo, jeśli implementacja będzie wystarczająco wydajna, porównane zostaną różne algorytmy liczenia średniej populacji, na przykład zwykła średnia i średnia ważona z wagami będącymi jakością punktów w populacji.

Dla każdej z metod wyliczania średniej oraz dla algorytmu niezmodyfikowanego każda z funkcji benchmarku CEC-2013 zostanie przetworzona k razy. Obliczone będą średnie wartości znalezionych optymalnych rozwiązań dla pewnych kroków czasowych. Pozwoli to wykreślić krzywe zbieżnoci algorytmów i praktycznie porównać ich działanie. Wartość k zostanie ustalona na etapie przygotowywania eksperymentów.

Ponadto, jeśli wyniki będą wystarczająco jednoznaczne, używając metod testowania hipotez statystycznych zostanie udowodniona wyższość jednej z metod.

\section{Podsumowanie}

Dla oceny algorytmu zmodyfikowanego kluczowe będą wyniki eksperymentów. Przed ich wykonaniem można przewidywać, że zmiana metody mutacji wpłynie na zmniejszenie ruchliwości i rozproszenia populacji. Wniosek ten wnika z faktu, że po zmianie, wszystkie punkty w zbiorze kandydatów będą rozmieszczone koncentrycznie wokoło punktu średniego populacji i oddalone od niego nie bardziej, niż wynosi maksymalna różnica punktów populacji.


\end{document}